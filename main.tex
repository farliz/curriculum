
% This is my curriculum vitae. It was prepared using ConTeXt text processing system.
% Lizardo Reyna | farliz@disroot.org
% July 2018

% This program is free software: you can redistribute it and/or modify
%     it under the terms of the GNU General Public License as published by
%     the Free Software Foundation, either version 3 of the License, or
%     (at your option) any later version.

%     This program is distributed in the hope that it will be useful,
%     but WITHOUT ANY WARRANTY; without even the implied warranty of
%     MERCHANTABILITY or FITNESS FOR A PARTICULAR PURPOSE.  See the
%     GNU General Public License for more details.

%     You should have received a copy of the GNU General Public License
%     along with this program.  If not, see <https://www.gnu.org/licenses/>.



\setuppapersize[A4][A4]
%\setuppagenumbering[state=stop]
\setupbodyfont[pagella, 10pt]
\usecolors[xwi]


\setuplayout[
  header=1.5cm,
  footer=1.5cm,
  height=fit,
  topspace=0.5cm,
  width=fit,
  backspace=1cm,
  rightmargin=0.5cm,
  leftmargin=0.5cm,
  leftmargindistance=0pt,
  rightmargindistance=0pt,
  grid=yes, columndistance=12pt,
columns=3]


% If you don't have some fonts, they can be replaced for your local fonts
\definefont[GaraB][name:garamondpremrprobdsubh*default sa 1.3]
\definefont[Skill][name:WhitneyHTF-BookSC*default sa 1.2]
\definefont[WhitneyB][name:whitneyhtfbold*default at 18pt]
\definefont[HeeboBl][name:heeboblack*default sa 1.5]
\definefont[Senes][name:WarnockPro-BoldItDisp*default sa 1]
\definefont[Facultad][name:WhitneyHTF-SemiBold*default sa 1]
\definefont[Fira][name:WhitneyHTF-Light*default sa 1]
\definefont[titleb][name:whitneyhtfbold*default sa 1.1]
\setuphead[subject][style=\HeeboBl]

\setupinteraction[state=start, color=darkblue]
\useURL[me][lreyna@utm.edu.ec]
\useURL[mee][reyna.utm@gmail.com]
\setupurl[color=darkslateblue, style=\Fira]

\setupfootertexts[\Fira Julio 2018][\Fira Lizardo Reyna~\cdot~Curriculum Vitae]

\setuppagenumbering[location={lower,middle}, style=\Fira]

\starttext

    \startTABLE[[background=color, frame=off]
    \bTABLE
    \bTR \bTD[align=middle,width=15cm ]
      {\WhitneyB Lizardo Mauricio Reyna Bowen}\\
  \color[darkgray]{\Skill   Física de suelos~\cdot~Geoinformación}
  
  \color[darkgray] {\Facultad Facultad de Ingeniería Agrícola, Universidad Técnica de Manabí}

        {\from[me] \| \from[mee] \| {\Fira + 593 982924637}}
        \eTD
        \bTD [align=right]{\externalfigure[yo_][width=2cm]}\\ \Fira C.I. 130835012-1\\ 19 Mayo 1983 \eTD\eTR
    \eTABLE
  \stopTABLE

\subject{Edu\color[darkgray]{cación}}

{\titleb Doctor en Ciencias de la Agronomía}\\
\Fira      
Facultad de Agronomía\\
Universidad de Concepción Chile \hfill 2017\\
{\Senes Registro en SENESCYT:   1521116770}

\blank[big]

{\titleb Master en Geoinformación}\\
\Fira
Centro de Levantamientos Aeroespaciales (CLASS)\\
Universidad Mayor de San Simón, Bolivia \hfill 2011\\
{\Senes Registro en SENESCYT:   5126R-11-11082}


\blank[big]

{\titleb Ingeniero Agrícola}\\
\Fira        
Facultad de Ingenierı́a Agrı́cola \\
Universidad Técnica de Manabı́, Portoviejo \hfill 2006\\
{\Senes Registro en SENESCYT:   1009-06-722453}

\blank[big]
\Fira
      {\titleb Complementario}\\
   Acreditado como Investigador Auxiliar 1 (SENESCYT)\\  
   Nivel B2 del idioma Ingles.\\
   Nivel avanzado en informática (GNU/Linux).


\subject{Experiencia \color[darkgray]{profesional}}

{\titleb Universidad Técnica de Manabí, Portoviejo}\hfill Marzo 2012 - Presente\\
Docente de la Facultad de Ingeniería Agrícola.\\
Dirección y participación en proyectos de investigación.\\
Participación en actividades de vinculación y capacitación.\\
Dirección de tesis de grado.


{\titleb Pontificia Universidad Católica del Ecuador, Portoviejo} \hfill Abril – Septiembre 2012\\
  Docente en la Carrera de Ingeniería Hidráulica.\\
  Docente del módulo de Hidrología.\\
  Dirección de tesis.

    
  {\titleb Organización de las Naciones Unidas para la Agricultura y la Alimentación
  (FAO)}\hfill Mayo – Octubre 2011\\
  Elaboración del estudio biofísico y sistema de información geográfica\\ de la cuenca Membrillo, Calceta,
  Manabí.


     
{\titleb Universidad Técnica Luis Vargas Torres}\hfill Varios Periodos\\
Docente en la Extensión de Muisne.\\
Docente del módulo de Ciencias Naturales.
     
{\titleb Vida Comercial Agrícola (VICOAGRI)} \hfill Abril – Octubre 2009\\
Técnico de campo, Manabí\\
      Diseño de sistemas de riego.\\
     Levantamientos topográficos.


     \subject{Proyectos de \color[darkgray]{investigación}}
     
\startitemize[packed]
\item Estudio Multitemporal del área cultivada de arroz y maíz en el valle del Río
  Portoviejo. Universidad Técnica de Manabí, 2012. 
\item Impacto del sistema del cultivo de arroz ({\em Oriza sativa}) sobre las propiedades
  del suelo en la zona arrocera de Charapotó. Universidad Técnica de Manabí, 2018
  
\stopitemize

\subject{Tesis \color[darkgray]{dirigidas}}

        {\bf Pregrado:}
\Fira        
\startitemize[packed]
\item Calibración de número de curva para el cálculo de abstracciones en condiciones de
  clima tropical. Pontificia Universidad Católica del Ecuador. Portoviejo, 2013. 
\item Caracterización hidrológica de la cuenca del río Garrapata utilizando modelos de
  elevación digital.  Pontificia Universidad Católica del Ecuador. Portoviejo, 2013. 
\item Elaboración de un sistema de información geográfico para la gestión del cultivo de
  arroz en la comunidades de San Bartolo y Bebedero, Cantón Sucre, Provincia de
  Manabí. Universidad Técnica de Manabí, 2013.
\item Evaluación de los modelos digitales de terreno SRTM y AsterGDEM para la extracción
  de cuencas hidrográficas en el valle del río Portoviejo. Universidad Técnica de Manabí,
  2018 --- continúa.
\stopitemize

  {\bf Maestría:}
  
\startitemize[packed]
\item Impacto del sistema del cultivo de arroz sobre las propiedades físicas y carbono orgánico
  del suelo en la zona arrocera de Charapotó. Universidad Técnica de Manabí, 2018  ---
  continúa.
\item Impacto de la mecanización en el cultivo de arroz sobre las propiedades físcas del
  suelo en la parroquia Charapotó. Universidad Técnica de Manabí, 2018 --- continúa.
\item Zonificación de suelos cultivados con Banano afectados por la acidez. Universidad
  Técnica de Manabí, 2018 --- continúa.
\stopitemize

\subject{Publi\color[darkgray]{caciones}}


        {\bf Artı́culos:}
 \Fira       
\startitemize[n, packed]
    \item Reyna-Bowen, L., Reyna-Bowen, M., Reina-Castro, J., \& Hurtado, E. (2017). Zonificación de uni-
      dades de paisaje en la cuenca del río Khora Tiquipaya (Quillacollo, Bolivia) sustentada en el enfoque
      de ecología del paisaje. La Técnica: Revista de Las Agrociencias. e-ISSN 2477-8982, 69–80.
    \item  Reyna-Bowen, L., Reyna-Bowen, M., \& Vera-Montengro, L. (2017). Zonificación del territorio para
      aplicar labranza de conservación mecanizada utilizando el enfoque de evaluación multicriterio. Revista
      Ciencias Técnicas Agropecuarias, 26(1), 40–49.
    \item Reyna, L., Dube, F., Barrera, J. A., \& Zagal, E. (2017). Potential Model Overfitting in Predicting
      Soil Carbon Content by Visible and Near-Infrared Spectroscopy. Applied Sciences,
      7(7).\\
      \goto{https://doi.org/10.3390/app7070708}[url(https://doi.org/10.3390/app7070708)]
    \item Castro, L. R., Reyna, L., \& Reyna-Bowen, L. (2014). Plan de desarrollo y ordenamiento territorial,
      como herramienta para el desarrollo sustentable: Estudio de caso en Esmeraldas, Ecuador. La Técnica,
      (12), 76–89.
    \item Reyna, L., \& Gossweiler, B. (2011). Segmentación y clasificación de imágenes satelitales para deter-
      minar la cobertura del suelo. La Técnica, (4), 38–44.
   \stopitemize

     {\bf Libros:}
\Fira
     \startitemize[n, packed, continue]

     \item Desarrollo y Ordenamiento Territorial Estudio de Caso Ecuador, Editorial Académica Española,
       ISBN-13: 978-3-8484-6175-2. 2014.
     \item Estudio Biofı́sico de la Cuenca Khora Tiquipaya, Editorial Académica Española,
       ISBN-13: 978-3-659-05060-2, 2014.
  \stopitemize

    {\bf Otras publicaciones:}
 \Fira   
\startitemize
  \item  El suelo, un recurso estratégico. El Diario Manabita, 11 de Marzo del 2016.
\stopitemize


\subject{Cursos \color[darkgray]{realizados}}

\startitemize[packed]
    \item La edafologı́a actual y sus aplicaciones. Universidad de Concepción, Chile 2017
      Hydrological and water management problems of semi-arid catchment: Floods, irrigation and water
      quality. Universidad de Concepción, Chile 2015.
    \item  Evaluación de impactos antropogénicos sobre áreas naturales con énfasis en patrones y procesos edáficos
      e hı́dricos. Universidad de Concepción, Chile 2014.
      Sistemas y tecnologı́as apropiadas en saneamiento. Instituto Federal Suizo de la Ciencia y Tecnologı́a
      del Agua. Atacames, Esmeraldas, 2013.
    \item   Sistema de Marco Lógico en la práctica. Universidad Técnica de Manabı́, Portoviejo 2013.
    \item   Seminario internacional en metodologı́a de la investigación. Universidad Tecnica de Manabı́. Porto-
      viejo, 2013.
    \item  Diseño de sistemas de riego presurizado aplicado a la agricultura urbana. Portoviejo, 2012.
    \item  Zonificación ecológica y económica. CISAS, FAO, CONCOPE, Lago Agrio, 2011.
    \item  Teledetección óptica y radiometrı́a. Comisión Nacional de Investigación y Desarrollo Aeroespacial.
      Perú 2008.
    \item  XX Curso de técnicas de riego y gestión del regadı́o. CENTER, Madrid, España, 2008.
    \item  Certificacion Forestal con Enfasis en Babu. INBAR, Portoviejo, 2008.
\stopitemize


\subject{Conferencias y \color[darkgray]{cursos impartidos}}

\startitemize[packed]

    \item Tecnologı́as para el estudio de suelos. Alternativas para impulsar el desarrollo agropecuario de Manabı́,
      después del 16 A. Universidad Técnica de Manabı́. Santa Ana, 2016
    \item  Clasificación de imágenes satelitales. Segundo diálogo nacional sobre Desertificación, Degradación de
      Tierras y Sequı́as. Ministerio del Ambiente, Chimborazo, 2012.
    \item  Sistemas de Información Geográfica. Universidad Técnica de Manabı́. Portoviejo, 2012
    \item  Sistemas de Posicionamiento Global. Técnica de Manabı́. Portoviejo, 2012
    \item  Teledetección como herramienta para el manejo de recursos naturales. Tercer diálogo nacional sobre
      Desertificación, Degradación de Tierras y Sequı́as. Portoviejo, 2012.
    \item Evaluación de inversiones para el sector agroalimentario de Ecuador, FAO, Quito, 2011.
    \item  Aplicación de sistemas de información geográfica y teledetección. Servicio Local de Acueductos y Al-
      cantarillados. Bolivia, 2010.
   \stopitemize


   \subject{Dignidades y \color[darkgray]{reconocimientos}}


   \startitemize[packed]
  \item Coodinador del Departamento de Ciencas Agrı́colas. Universidad Técnica de Manabı́, 2018.
  \item    Reconocimiento por haber culminado los estudios doctorales. Colegio Regional de Ingenieros Agrı́colas
     del Litoral. Portoviejo, 2018.
   \item   Reconocimiento al primer graduado del programa de Doctorado en Ciencias de la Agronomı́a, Universidad de Concepción, Chile, 2017.
   \stopitemize


   \subject{Otras \color[darkgray]{actividades}}

   \startitemize[packed]
       \item  Evento científico {\em Científicos en la Rotonda}, Portoviejo, Noviembre 2017.
       \item  Editor de la revista {\em International Journal of College and University}, 2017.
       \item  Evento de difusión científica. {\em Feria nacional de Ciencia y Tecnología}. Quito, Diciembre 2009.
       \stopitemize

       
%        \vfill
%        \switchtobodyfont[9pt]
%        \Fira
% \hfill \color[darkgray]{Este documento fué creado con \CONTEXT ~~ \contextversion}
\stoptext

