
% This is my curriculum vitae. It was prepared using ConTeXt text processing system.
% Lizardo Reyna | farliz@disroot.org
% July 2018

% This program is free software: you can redistribute it and/or modify
%     it under the terms of the GNU General Public License as published by
%     the Free Software Foundation, either version 3 of the License, or
%     (at your option) any later version.

%     This program is distributed in the hope that it will be useful,
%     but WITHOUT ANY WARRANTY; without even the implied warranty of
%     MERCHANTABILITY or FITNESS FOR A PARTICULAR PURPOSE.  See theiq
%     GNU General Public License for more details.

%     You should have received a copy of the GNU General Public License
%     along with this program.  If not, see <https://www.gnu.org/licenses/>.



\setuppapersize[A4][A4]
%\setuppagenumbering[state=stop]
\setupbodyfont[pagella, 11pt]
\usecolors[xwi]
\definecolor[cdiez][x=969696]
\usesymbols[fontawesome]
\setuplayout[
  header=1.5cm,
  footer=1.5cm,
  height=fit,
  topspace=0.5cm,
  width=fit,
  backspace=2cm,
  rightmargin=.5cm,
  leftmargin=1cm,
  leftmargindistance=0pt,
  rightmargindistance=0pt,
  grid=yes, columndistance=12pt,
columns=3]


% If you don't have some fonts, they can be replaced for your local fonts
\definefont[GaraB][name:garamondpremrprobdsubh*default sa 1.3]
\definefont[Skill][name:WhitneyHTF-BookSC*default sa 1.2]
\definefont[WhitneyB][name:whitneyhtfbold*default at 22pt]
\definefont[HeeboBl][name:heeboblack*default sa 1.5]
\definefont[Senes][name:WhitneyHTF-SemiBoldItalic*default sa 1]
\definefont[Facultad][name:WhitneyHTF-SemiBold*default sa 1.2]
\definefont[Body][name:WhitneyHTF-Light*default sa 1]
\definefont[BodyB][name:WhitneyHTF-SemiBoldSC*default sa 1]
\definefont[Mail][name:FiraSans-Medium*default sa 1]
\definefont[titleb][name:whitneyhtfbold*default sa 1.1]

\setuphead[subject][style=\HeeboBl]

\setupinteraction[state=start, color=darkblue]
\useURL[me][lreyna@utm.edu.ec]
\useURL[mee][reyna.utm@gmail.com]
\setupurl[color=darkslateblue, style=\Mail]

\setupfootertexts[\Body Abril 2020][\Body Lizardo Reyna~\cdot~Curriculum Vitae]

\setuppagenumbering[location={lower,middle}, style=\Body]

\starttext

    \startTABLE[[background=color, frame=off]
    \bTABLE
    \bTR \bTD[align=right,width=14cm ]
      {\WhitneyB Lizardo Mauricio Reyna Bowen}\\
  \color[darkgray]{\Skill   Física de suelos~\cdot~Geoinformación}
  
  \color[darkgray] {\Facultad Facultad de Ingeniería Agrícola, Universidad Técnica de Manabí}

       \symbol[fontawesome][envelope] {\from[me] \|  \symbol[fontawesome][envelope] \from[mee] \| \symbol[fontawesome][mobile] {\Mail +593 982924637}}
        \eTD
        \bTD [align={lohi}] {\externalfigure[yo2019bw][width=3.5cm]}\\ \goto{https://lreyna.com/}[url(https://lreyna.com/)]  \eTD\eTR
    \eTABLE
  \stopTABLE

  %\blank[2*big]

  \startTABLE[[background=color, frame=off]
    \bTABLE
    \bTR 
        \bTD {\em 
Doctor en ciencias de la Agronomía con más de ocho años de experiencia en docencia e
investigación científica. Utilizo sistemas de información geográfica, teledetección y
espectroscopía para el estudio de los suelos y la vegetación. Promuevo el uso de software
libre para el análisis de datos, preparación de documentos de texto y
publicaciones. Trabajo con entusiasmo, me adapto al trabajo colaborativo y a la
utilización de nuevas tecnologías.
  }  \eTD\eTR
    \eTABLE
  \stopTABLE

  

%%%%%%%%%%%%%%%%%%%%%%%%%%%%%%%%%%%%%%%%%%%%%%%%%%%%
%%%%%%%%%%%%%%%%% Educación %%%%%%%%%%%%%%%%%%%%%%%%
\subject{Edu\color[darkgray]{cación}\thinrule}

{\titleb Doctor en Ciencias de la Agronomía}\\
\Body     
Facultad de Agronomía\\
Universidad de Concepción Chile \hfill 2017\\
{\Senes Registro en SENESCYT:   1521116770}

\blank[big]

{\titleb Master en Geoinformación}\\
Centro de Levantamientos Aeroespaciales (CLASS)\\
Universidad Mayor de San Simón, Bolivia \hfill 2011\\
{\Senes Registro en SENESCYT:   5126R-11-11082}


\blank[big]

{\titleb Ingeniero Agrícola}\\       
Facultad de Ingenierí Agrícola \\
Universidad Técnica de Manabí, Portoviejo \hfill 2006\\
{\Senes Registro en SENESCYT:   1009-06-722453}

\blank[1.5cm]


\startnarrower
      {\titleb Complementario}
      \blank[medium]
{\Mail Acreditado como Investigador Auxiliar 1}---
{\Senes resgistro en SENESCYT: REG-INV-18-03167}\\  
Nivel B2 del idioma Ingles.\\
Nivel avanzado en informática (GNU/Linux + software libre).
% \blank[medium]
% {\titleb Grupos y membresías}

% Free Software Foundation, www.fsf.org {\bf Associate Member}.

\stopnarrower

%%%%%%%%%%%%%%%%%%%%%%%%%%%%%%%%%%%%%%%%%%%%%%%%%%%%
%%%%%%%%%%%%%%%%% Experiencia Laboral %%%%%%%%%%%%%%
   
\subject{Experiencia \color[darkgray]{profesional}\thinrule}

{\titleb Universidad Técnica de Manabí, Portoviejo}\hfill Marzo 2012 - Presente\\
Docente de la Facultad de Ingeniería Agrícola.\\
Dirección y participación en proyectos de investigación.\\
Participación en actividades de vinculación y capacitación.\\
Dirección de tesis de grado.

\blank

{\titleb Pontificia Universidad Católica del Ecuador, Portoviejo} \hfill Abril – Septiembre 2012\\
  Docente en la Carrera de Ingeniería Hidráulica.\\
  Docente del módulo de Hidrología.\\
  Dirección de tesis.

\blank
  
  {\titleb Organización de las Naciones Unidas para la Agricultura y la Alimentación
  (FAO)}\hfill Mayo – Octubre 2011\\
  Elaboración del estudio biofísico y sistema de información geográfica\\ de la cuenca Membrillo, Calceta,
  Manabí.


     
{\titleb Universidad Técnica Luis Vargas Torres}\hfill Varios Periodos\\
Docente en la Extensión de Muisne.\\
Docente del módulo de Ciencias Naturales.

\blank

{\titleb Vida Comercial Agrícola (VICOAGRI)} \hfill Abril – Octubre 2009\\
Técnico de campo, Manabí\\
      Diseño de sistemas de riego.\\
     Levantamientos topográficos.


%%%%%%%%%%%%%%%%%%%%%%%%%%%%%%%%%%%%%%%%%%%%%%%%%%%%
%%%%%%%%%%%%%%%%% Investigación %%%%%%%%%%%%%%%%%%%%
\subject{Proyectos de \color[darkgray]{investigación}\thinrule}
     
    \startitemize[packed]
        \item Estudio Multitemporal del área cultivada de arroz y maíz en el valle del Río
          Portoviejo. Universidad Técnica de Manabí, 2012. 
        \item Impacto del sistema del cultivo de arroz ({\em Oriza sativa}) sobre las propiedades
          del suelo en la zona arrocera de Charapotó. Universidad Técnica de Manabí, 2018
    \stopitemize


%%%%%%%%%%%%%%%%%%%%%%%%%%%%%%%%%%%%%%%%%%%%%%%%%%%%
%%%%%%%%%%%%%%%%% Tesis dirigidas %%%%%%%%%%%%%%%%%%
\subject{Tesis \color[darkgray]{dirigidas}\thinrule}

        {\titleb Pregrado:}
\Body       
\startitemize[packed]
  \item Evaluación de los modelos digitales de terreno SRTM y AsterGDEM para la extracción
    de cuencas hidrográficas en el valle del río Portoviejo. Universidad Técnica de Manabí,
    2018.
  \item Análisis de la variabilidad espacial de la
    compactación de suelo en una parcela experimental del
    Campus Lodana, Universidad Técnica de Manabí, 2018
    
  \item Calibración de número de curva para el cálculo de abstracciones en condiciones de
    clima tropical. Pontificia Universidad Católica del Ecuador. Portoviejo, 2013. 
  \item Caracterización hidrológica de la cuenca del río Garrapata utilizando modelos de
    elevación digital.  Pontificia Universidad Católica del Ecuador. Portoviejo, 2013. 
  \item Elaboración de un sistema de información geográfico para la gestión del cultivo de
    arroz en la comunidades de San Bartolo y Bebedero, Cantón Sucre, Provincia de
    Manabí. Universidad Técnica de Manabí, 2013.
            
    \stopitemize

  {\titleb Maestría:}
  
\startitemize[packed]
\item Impacto del sistema del cultivo de arroz sobre las propiedades físicas y carbono orgánico
  del suelo en la zona arrocera de Charapotó. Maestría en Ingeniería Agrícola,  Universidad Técnica de Manabí, 2018 .
\item Impacto de la mecanización en el cultivo de arroz sobre las propiedades físcas del
  suelo en la parroquia Charapotó. Maestría en Ingeniería Agrícola,  Universidad Técnica de Manabí, 2018.
\item Zonificación de suelos cultivados con Banano afectados por la acidez. Maestría en Ingeniería Agrícola, Universidad
  Técnica de Manabí, 2018.
\stopitemize


%%%%%%%%%%%%%%%%%%%%%%%%%%%%%%%%%%%%%%%%%%%%%%%%%%%%
%%%%%%%%%%%%%%%%% Publicaciones  %%%%%%%%%%%%%%%%%%%
\subject{Publi\color[darkgray]{caciones}\thinrule}


        {\titleb Artículos:}
       
        \startitemize[n, packed]

        \item Reyna-Bowen L, Vera-Montenegro L, {\bf Reyna L.}
          Soil-Organic-Carbon Concentration and Storage under
          Different Land Uses in the Carrizal-Chone Valley in
          Ecuador. {\it Applied Sciences}. 2019;
          9(1):45.\\ \goto{https://doi.org/10.3390/app7070708}[url(tps://doi.org/10.3390/app7070708)]
          
        \item Reyna-Bowen, L., {\BodyB Reyna-Bowen, M.}, Reina-Castro,
          J., \& Hurtado, E. (2017). Zonificación de unidades de
          paisaje en la cuenca del río Khora Tiquipaya (Quillacollo,
          Bolivia) sustentada en el enfoque de ecología del
          paisaje. {\it La Técnica: Revista de Las Agrociencias}. e-ISSN
          2477-8982, 69–80.\\
          \goto{https://dialnet.unirioja.es/servlet/articulo?codigo=6258737}
               [url(https://dialnet.unirioja.es/servlet/articulo?codigo=6258737)]
  
        \item Reyna-Bowen, L., {\BodyB Reyna-Bowen, M.}, \&
          Vera-Montengro, L. (2017). Zonificación del territorio para
          aplicar labranza de conservación mecanizada utilizando el
          enfoque de evaluación multicriterio. {\it Revista Ciencias
          Técnicas Agropecuarias}, 26(1), 40–49.\\
          \goto{http://scielo.sld.cu/scielo.php?script=sci_arttext&pid=S2071-00542017000100005}
          [url(http://scielo.sld.cu/scielo.php?script=sci_arttext&pid=S2071-00542017000100005)]
      
        \item {\BodyB Reyna, L.}, Dube, F., Barrera, J. A., \& Zagal,
          E. (2017). Potential Model Overfitting in Predicting Soil
          Carbon Content by Visible and Near-Infrared
          Spectroscopy. {\it Applied Sciences},
          7(7).\\ \goto{https://doi.org/10.3390/app7070708}[url(https://doi.org/10.3390/app7070708)]
      
        \item Castro, L. R., {\BodyB Reyna, L.}, \& Reyna-Bowen,
          L. (2014). Plan de desarrollo y ordenamiento territorial,
          como herramienta para el desarrollo sustentable: Estudio de
          caso en Esmeraldas, Ecuador. {\it La Técnica}, (12), 76–89.\\
          \goto{https://dialnet.unirioja.es/servlet/articulo?codigo=6087661}
          [url(https://dialnet.unirioja.es/servlet/articulo?codigo=6087661)]
      
        \item {\BodyB Reyna, L.}, \& Gossweiler,
          B. (2011). Segmentación y clasificación de imágenes
          satelitales para determinar la cobertura del suelo. {\it La
          Técnica}, (4), 38–44.\\
          \goto{https://revistas.utm.edu.ec/index.php/latecnica/article/view/663}
          [url(https://revistas.utm.edu.ec/index.php/latecnica/article/view/663)]
      
   \stopitemize

     {\titleb Libros:}

     \startitemize[n, packed, continue]

     \item Sustentabilidad de los sistemas agrpecuarios del Carrizal-Chone. Editorial Académica Española\\
       ISBN-13 978-620-2-12596-3, 2018.

     \item Desarrollo y Ordenamiento Territorial Estudio de Caso Ecuador, Editorial Académica Española\\
       ISBN-13: 978-3-8484-6175-2. 2014.
     \item Estudio Biofísico de la Cuenca Khora Tiquipaya, Editorial Académica Española,
       ISBN-13: 978-3-659-05060-2, 2014.
  \stopitemize


    {\titleb Otras publicaciones:}
   
\startitemize
    \item Sustentabilidad de los sistemas agropecuarios en la zona del proyecto de riego
      Carrizal-Chone en Manabí, Ecuador. Primer congreso internacional sobre {\BodyB
        Eficiencia en el uso del agua para la producción
        agrícola}. INACAP 2017, Chile.
      
    \item Potencial sobreajuste de modelos matemáticos en la
      predicción de carbono orgánico del suelo por
      espectroscopía. {\BodyB Primera Convención Científica}. Universidad
      Técnica de Manabí, ISBN:  978-9942-948-14-4, Octubre 2017.

      \item Matplotlib for Climate Change. Script en Python y Jupyter Notebook. \goto{Link}[url(https://nbviewer.jupyter.org/github/farliz/MatplotlibForClimateChange/blob/master/MatplotlibForClimateChange-R.ipynb)]

    \item El suelo, un recurso estratégico. {\BodyB El Diario Manabita}, 11 de Marzo del 2016.
  
    
\stopitemize


%%%%%%%%%%%%%%%%%%%%%%%%%%%%%%%%%%%%%%%%%%%%%%%%%%%%
%%%%%%%%%%%%%%%%% Cursos Realizados %%%%%%%%%%%%%%%%
\subject{Cursos \color[darkgray]{realizados}\thinrule}

\startitemize[packed]
  \item Linear Regression with Nunpy and Python. Coursera (online), Abril 2020.
  \item Curso de Agricultura de precisión mediante DRON EBEE SQ y Procesamiento con
    PIX4D. EsinteGeo s.a. julio 2019 (30 horas).
  \item Curso Internacional de Radar de Apertura Sintética (SAR). Geoespace Solutions
    Quito, junio 2019 (40 horas)
 \item Curso básico sobre métodos estadísticos. Universidad Técnica de Manabí, Febrero
   2019 (40 horas)
     \item Razonamiento estadístico y análisis de datos. Escuela Sueperior Politécnica de Manabí ESPAM MFL (80 horas). Diciembre 2018. 
    \item Diseño y desarrollo de aulas virtuales en la UTM. Universidad Técnica de Manabí, Ecuador 2018. 
    \item La edafología actual y sus aplicaciones. Universidad de Concepción, Chile 2017
    \item Hydrological and water management problems of semi-arid catchment: Floods, irrigation and water
      quality. Universidad de Concepción, Chile 2015.
    \item  Evaluación de impactos antropogénicos sobre áreas naturales con énfasis en patrones y procesos edáficos
      e hídricos. Universidad de Concepción, Chile 2014.
      \item Sistemas y tecnologías apropiadas en saneamiento. Instituto Federal Suizo de la Ciencia y Tecnología
      del Agua. Atacames, Esmeraldas, 2013.
    \item   Sistema de Marco Lógico en la práctica. Universidad Técnica de Manabí, Portoviejo 2013.
    \item   Seminario internacional en metodología de la investigación. Universidad Técnica de Manabí. Portoviejo, 2013.
    \item  Diseño de sistemas de riego presurizado aplicado a la agricultura urbana. Portoviejo, 2012.
    \item  Zonificación ecológica y económica. CISAS, FAO, CONCOPE, Lago Agrio, 2011.
    \item  Teledetección óptica y radiometría. Comisión Nacional de Investigación y Desarrollo Aeroespacial. Perú, 2008.
    \item  XX Curso de técnicas de riego y gestión del regadío. CENTER, Madrid, España, 2008.
    \item  Certificación Forestal con Énfasis en Bambú. INBAR, Portoviejo, 2008.
\stopitemize


%%%%%%%%%%%%%%%%%%%%%%%%%%%%%%%%%%%%%%%%%%%%%%%%%%%%
%%%%%%%%%%%%%%%%% Conferencias %%%%%%%%%%%%%%%%%%%%&
\subject{Conferencias y \color[darkgray]{cursos impartidos}\thinrule}

\startitemize[packed]

\item Uso de imágenes Sentinel-2 para la evlauación de la vegetación: aplicaciones y limitaciones.\\ {\it XI Congreso Latinoamericano de Agronomía} (online). ISBN: 978-9942-802-58-3, CIDE, Abril 2020.

\item Estimación de la erosividad en la Provincia de Manabí con el método CORINE. \\ {\it XI Congreso Internacional de Investigación Científica e innovación Tecnológica} CIICIT. Universidad Laica Eloy Alfaro de Manabí ULEAM, Febrero 2020.

  \item Impacto del sistema deo cultivo de arroz sobre las propiedades
    del suelo.\\ {\it Tercera Convención Científica UTM}, Universidad Técnica
    de Manabí, Octubre 2019.

   \item Procesamiento de imágenes para la estimación de la biomasa
    vegetal. Primera sesioń científica del proyecto {\it Una Geoweb para
     el Desarrollo Sostenible}. Universidad Técnica de Manabí, 2018.

    \item Tecnologías para el estudio de suelos. \\{\it Alternativas para impulsar el desarrollo agropecuario de Manabí,
      después del 16 A}. Universidad Técnica de Manabí. Santa Ana, 2016
    \item  Clasificación de imágenes satelitales. \\{\it  Segundo diálogo nacional sobre Desertificación, Degradación de
      Tierras y Sequías}. Ministerio del Ambiente, Chimborazo, 2012.
    \item  Curso de {\it Sistemas de Información Geográfica}. Universidad Técnica de Manabí. Portoviejo, 2012
    \item  Curso de {\it Sistemas de Posicionamiento Global}. Universidad Técnica de Manabí. Portoviejo, 2012
    \item  Teledetección como herramienta para el manejo de recursos naturales. \\{\it Tercer diálogo nacional sobre
      Desertificación, Degradación de Tierras y Sequías}. Portoviejo, 2012.
    \item Evaluación de inversiones para el sector agroalimentario de Ecuador, {\it FAO}, Quito, 2011.
    \item  Aplicación de sistemas de información geográfica y teledetección. {\it Servicio Local de Acueductos y Alcantarillados}. Bolivia, 2010.
   \stopitemize


%%%%%%%%%%%%%%%%%%%%%%%%%%%%%%%%%%%%%%%%%%%%%%%%%%%%
%%%%%%%%%%%%%%%%% Dignidades %%%%%%%%%%%%%%%%%%%%%%%%
   \subject{Dignidades y \color[darkgray]{reconocimientos}\thinrule}


   \startitemize[packed]
  \item Coodinador del Departamento de Ciencas Agrícolas. Universidad Técnica de Manabí, 2018.
  \item    Reconocimiento por haber culminado los estudios doctorales. Colegio Regional de Ingenieros Agrícolas
     del Litoral. Portoviejo, 2018.
   \item   Reconocimiento al primer graduado del programa de Doctorado en Ciencias de la Agronomı́a, Universidad de Concepción, Chile, 2017.
   \stopitemize


%%%%%%%%%%%%%%%%%%%%%%%%%%%%%%%%%%%%%%%%%%%%%%%%%%%%
%%%%%%%%%%%%%%%%% Otras %%%%%%%%%%%%%%%%%%%%%%%%%%%%
   \subject{Otras \color[darkgray]{actividades}\thinrule}

   \startitemize[packed]
       \item  Evento científico  Científicos en la Rotonda, Portoviejo, Noviembre 2017.
      % \item  Editor de la revista International Journal of College and University, 2017.
       \item  Evento de difusión científica.  Feria Nacional de Ciencia y Tecnología. Quito, Diciembre 2009.
       \stopitemize

       
 \vfill     
 \switchtobodyfont[termes, 9pt]
 
 \startalignment[left]
 \color[darkgray]{ \CONTEXT ~~ \contextversion}\\
  \color[darkgray]{GNU/Linux}
\stopalignment

\stoptext

